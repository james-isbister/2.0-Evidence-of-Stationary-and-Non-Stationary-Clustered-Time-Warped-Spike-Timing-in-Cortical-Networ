
%<*ClusterCorrelationSummary0>
Spike pairs within Stage 1 cluster ellipses were linearly correlated above chance levels (Fisher's method: $\chi^{2}(7812, N = 3906) = 15150.9, $ $p < .001$; Fig. 3e). 
%</ClusterCorrelationSummary0>

%<*ClusterCorrelationSummary0p1>
Moreover, for Stage 2 clusters, spike pair differences were correlated with the spike time of the first unit in the pair (Fisher's method: $\chi^{2}(324, N = 162) = 7832.7, $ $p < .001$; Fig. 3g), showing that correlations were not explained by a model of the form $s_1 = s_0 + d + independent\_noise$ where $d$ is a fixed difference.
%</ClusterCorrelationSummary0p1>

%<*ClusterCorrelationSummary1>
97.5\% (158/162) of Stage 2 clusters were positively correlated (Fig. 3f). 
%</ClusterCorrelationSummary1>

%<*ClusterCorrelationSummary2>
80.9\% (131/162) of the Stage 2 clusters had correlations of higher significance than correlations calculated for all unclustered first spike pairs in the original response distribution (Fig. 3h). Moreover, 59.9\% (97/162) of the original response distributions from which Stage 2 clusters were extracted were not correlated significantly (p>0.05) (Fig. 3h). 
%</ClusterCorrelationSummary2>

%<*AngledClustersUniquePairsSummary>
A total of 162 unique Stage 2 clusters were extracted from 162 unique response distributions.
%</AngledClustersUniquePairsSummary>

%<*OriginalAnglesSummary>
67.9\% (110/162) of $\theta_{45}$ angles were both significantly less than $45^{\circ}$ (p<0.025) and greater than $0^{\circ}$ (p<0.025). 
%</OriginalAnglesSummary>

%<*OriginalTestsPassedSummary>
46 Stage 2 clusters were determined as stationary and correlated (see Methods \ref{methods:angle_and_factor_analysis_criteria} for criteria).
%</OriginalTestsPassedSummary>

%<*OriginalTestsPassedAngleSummary>
30 stationary Stage 2 clusters had estimated angles significantly different from both $45^{\circ}$ (p<0.025; Fig. 4f) and $0^{\circ}$ (p<0.025; Fig. 4f). 
%</OriginalTestsPassedAngleSummary>

%<*OriginalTestsPassedAndNormalAngleSummary>
Normality was not rejected for 20 stationary Stage 2 clusters (Henze-Zirkler p > 0.05), of which 14 had an estimated angle significantly different from both $45^{\circ}$ (p-value < 0.025) and $0^{\circ}$ (p-value < 0.025).
%</OriginalTestsPassedAndNormalAngleSummary>

%<*OriginalTestsPassedFactorAnalysisSummary>
As each Stage 2 cluster fulfilled the criteria of Methods \ref{methods:angle_and_factor_analysis_criteria}, factor analysis could be applied to the stationary Stage 2 clusters to give an estimate of state-conditioned response distributions for each cluster.
%</OriginalTestsPassedFactorAnalysisSummary>

%<*ClusterSingleUnitKPSSStationaritySummary>
For the 162 angled clusters, 117 of the corresponding 324 ($=2*162$) single unit cluster first spike sequences were determined as KPSS non-stationary (p < 0.05; Fig. 4d).
%</ClusterSingleUnitKPSSStationaritySummary>

%<*ClusterSingleUnitTrialIndexLRSummary>
The cluster spike times of either 0, 1 or 2 of the units in a cluster pair were correlated with trial index (Fisher's method: $\chi^{2}(648, N = 324) = 2485.1, $ $p < .001$; Fig. 4b,c).
%</ClusterSingleUnitTrialIndexLRSummary>

%<*SelectivelyDifferencedSummary>
In total, 83 clusters had differencing applied to atleast one unit. 83 of these clusters fulfilled the criteria, giving a total of 83 criteria fulfilling angled clusters (including the undifferenced stationary clusters). 
%</SelectivelyDifferencedSummary>

%<*SelectivelyDifferencedCorrelatedSummary>
In total, 53 of the 83 criteria fulfilling clusters were linearly correlated following differencing (p-value < 0.05) (Fig. 6h). 23 of the correlated criteria fulfilling clusters had estimated angles significantly greater than $0^{\circ}$ (p-value < 0.025) and less than $45^{\circ}$ (p-value < 0.025). Of these, normality was not rejected for 13 clusters (Henze-Zirkler p > 0.05). Fig. 6j shows that the distribution of estimated angles for the successfully differenced clusters. Fig. 8? shows the difference between original and differenced estimated clusters?
%</SelectivelyDifferencedCorrelatedSummary>

%<*SelectivelyDifferencedTestsPassedNewSummary>
In total, 83 clusters had differencing applied to atleast one unit. 35 of these were correlated and fulfilled the criteria. 23 of these clusters had estimated angles significantly greater than $0^{\circ}$ (p-value < 0.025; Fig 6b) and less than $45^{\circ}$ (p-value < 0.025; Fig 6b), of which 12 normality was not rejected for (Henze-Zirkler p < 0.05). These angles are similar to the original angles before differencing (Fig. 6c), showing that the differencing operation does not largely alter cluster angle. Fig. 6d shows the amount of variance explained by the pairwise correlation after trend removal. 
%</SelectivelyDifferencedTestsPassedNewSummary>

%<*SelectivelyDifferencedTestsPassedEigenvalueSummary>
The factor analysis model from above can be applied to the 35 correlated criteria fulfilling differenced clusters, as the factor correlation matrices of each cluster only had one eigenvalue greater than 1. 
%</SelectivelyDifferencedTestsPassedEigenvalueSummary>

%<*SelectivelyDifferencedBoxJenkinsDifferencedSummary>
81 of the 83 differenced clusters had an ARMA model applied to atleast one unit. Almost all ARMA models fit for differenced clusters were of the MA type (Fig 7a), as expected for the negative autocorrelations introduced by differencing. 
%</SelectivelyDifferencedBoxJenkinsDifferencedSummary>

%<*SelectivelyDifferencedBoxJenkinsCorrelatedSummary>
After the described application of the Box-Jenkins method, there were 42 criteria fulfilling clusters. Normality was not rejected for 18 of these clusters (Henze-Zirkler p > 0.05). 27 of the correlated criteria fulfilling clusters had estimated angles significantly greater than $0^{\circ}$ (p-value < 0.025; Fig. 7b) and less than $45^{\circ}$ (p-value < 0.025; Fig. 7b), for which 10 normality was not rejected for (Henze-Zeckler p-value > 0.05). Fig. 7c shows the difference between original and ARIMA estimated cluster angles. 
%</SelectivelyDifferencedBoxJenkinsCorrelatedSummary>

%<*SelectivelyDifferencedBoxJenkinsTestsPassedEigenvalueSummary>
The factor analysis model from above can be applied to the 42 correlated criteria fulfilling clusters, as the factor correlation matrices of each cluster only had one eigenvalue greater than 1. 
%</SelectivelyDifferencedBoxJenkinsTestsPassedEigenvalueSummary>

